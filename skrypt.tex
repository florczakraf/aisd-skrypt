\documentclass[10pt,b5paper]{book}
\usepackage{polski}
\usepackage{amsthm}
\usepackage[T1]{fontenc}
\usepackage[utf8]{inputenc}
\usepackage{geometry}
\usepackage{tikz}
\usepackage{amsmath}
\usepackage{amssymb}
\usepackage{array}
\usepackage[titletoc]{appendix}
\usepackage{scrextend}
\usepackage[ruled,commentsnumbered]{algorithm2e}
\usepackage{adjustbox}
\usepackage[hidelinks]{hyperref}

%% Biblioteki TiKZa

\usetikzlibrary{shapes.arrows}

%% Użyteczne komendy

\newtheorem{theorem}{Twierdzenie}
\newtheorem{definition}{Definicja}
\newtheorem{lemma}{Lemat}
\newtheorem{fact}{Fakt}
\newtheorem{observation}{Obserwacja}
\def\checkmark{\tikz\fill[scale=0.4](0,.35) -- (.25,0) -- (1,.7) -- (.25,.15) -- cycle;} 
\newcommand{\tizkboxwithcaption}[2]{\noindent\begin{figure}[!h]\centering\begin{adjustbox}{min width=0.75\textwidth, max width=1\textwidth}\input{#1}\end{adjustbox}\caption{#2}\end{figure}}

\newcommand{\comment}[1]{\marginpar{\tiny\raggedright #1}}

\renewcommand{\KwData}{\textbf{Input:}}
\renewcommand{\KwResult}{\textbf{Output:}}

%% Strona tytułowa

\usepackage[load-headings]{exsheets}
\DeclareInstance{exsheets-heading}{mylist}{default}{
  runin = true ,
  attach = {
    main[l,vc]number[l,vc](-3em,0pt) ;
    main[r,vc]points[l,vc](\linewidth+\marginparsep,0pt)
  }
}

\SetupExSheets{
  headings = mylist ,
  headings-format = \normalfont ,
  counter-format = se.qu ,
  counter-within = section
}

\usepackage{etoolbox}
\AtBeginEnvironment{question}{\addmargin[3em]{0em}}
\AtEndEnvironment{question}{\endaddmargin}

\newcommand*{\titleGM}{\begingroup
\hbox{
\hspace*{0.2\textwidth}
\hspace*{0.05\textwidth}
\parbox[b]{0.75\textwidth}{
{\noindent\Huge\bfseries Skrypt \\[0.2\baselineskip] z Algorytmów \\[0.2\baselineskip] i~struktur danych}\\[2\baselineskip]
{\large \textit{Zbiór mniej lub bardziej ciekawych algorytmów i~struktur danych, jakie bywały omawiane na wykładzie (albo i nie).}}\\[3\baselineskip]
{\Large \textsc{praca zbiorowa pod redakcją \\[0.1\baselineskip] Krzysztofa Piecucha}}

\vspace{0.5\textheight}
{\noindent Korzystać na własną odpowiedzialność.}\\[\baselineskip]
}}
\endgroup}

\definecolor{titlepagecolor}{cmyk}{0.9,1,.6,.40}

\newcommand\titlepagedecoration{%
\begin{tikzpicture}[remember picture,overlay,shorten >= -10pt]

\coordinate (aux1) at ([yshift=-15pt]current page.north west);
\coordinate (aux2) at ([yshift=-450pt]current page.north west);
\coordinate (aux3) at ([xshift=+4.5cm]current page.north west);
\coordinate (aux4) at ([yshift=-150pt]current page.north west);

\begin{scope}[titlepagecolor!60,line width=14pt,rounded corners=12pt]
\draw
  (aux1) -- coordinate (a)
  ++(-45:5) --
  ++(225:5.1) coordinate (b);
\draw[shorten <= -10pt]
  (aux3) --
  (a) --
  (aux1);
\draw[opacity=0.4,titlepagecolor,shorten <= -10pt]
  (b) --
  ++(-45:2.2) --
  ++(225:2.2);
\end{scope}
\draw[titlepagecolor,line width=9pt,rounded corners=8pt,shorten <= -10pt]
  (aux4) --
  ++(-45:1.2) --
  ++(225:1.2);
\begin{scope}[titlepagecolor!70,line width=6pt,rounded corners=6pt]
\draw[shorten <= -10pt]
  (aux2) --
  ++(-45:3) coordinate[pos=0.45] (c) --
  ++(225:3.1);
\draw
  (aux2) --
  (c) --
  ++(45:2.5) --
  ++(135:2.5) --
  ++(225:2.5) coordinate[pos=0.3] (d);   
\draw 
  (d) -- +(135:1);
\end{scope}
\end{tikzpicture}%
}

\begin{document}
\pagenumbering{arabic}
\pagestyle{empty}

\titleGM
\titlepagedecoration

\tableofcontents 
\pagestyle{plain}

\chapter{Zrobione}

\include{mastertheorem}

\include{bitoniczne}

\include{fibonacci}

\include{strassen}

\include{elementuniqness}

\chapter{Under construction}

\include{zlozonosc}

\include{modelobliczen}

\include{kopiec}

\include{wiesniakow}

\include{sortowanietopologiczne}

\include{sortowanie}

\include{mst}

\include{dijkstra}

\include{szeregowanie}

\include{dynamicznenadrzewach}

\include{plecaki}

\include{problemynp}

\include{siecibenesa}

\section{Haszowanie}

W tym rozdziale zastanowimy się jak sprawnie zrealizować słownik --- strukturę danych udostępniającą operacje: 
\begin{itemize}
	\item dodawania kluczy,
	\item usuwania kluczy,
	\item sprawdzania czy klucz jest w słowniku (i zwracania przypisanej do niego wartości).
\end{itemize}
Przyjmijmy, że wszystkie klucze należą do pewnego uniwersum $U$.

Do wykonywania powyższych operacji możemy użyć drzew zbalansowanych (np. AVL lub czerwono-czarnych), jednak w tym przypadku będziemy działać w czasie $O(log(n))$. Poniżej przedstawimy w jaki sposób można zejść do zamortyzowanego czasu stałego.

\subsection{Tablice z adresowaniem bezpośrednim}
Zastanówmy się przez chwilę, jak można zrealizować opisane wcześniej operacje posiadając nieograniczoną pamięć --- albo inaczej: $U$ jest na tyle małe, że mieści się w pamięci, np. $U = \{0, 1, \dots, m - 1\}$. Dodatkowo załóżmy, że żadne dwa elementy nie mają identycznych kluczy.

Przy takich założeniach, pewien podzbiór kluczy z $U$ możemy reprezentować przy użyciu zwykłej tablicy $T[0 \dots m - 1]$. Jeśli klucz $k$ znajduje się w $T$, to w $T[k]$ znajduje się wskaźnik na element przypisany do $k$; w przeciwnym wypadku $T[k] = NIL$.

\subsection{Tablice haszujące}
Opisane w poprzednim podrozdziale rozwiązanie wymaga nieakceptowalnego założenia --- nieograniczonej pamięci. Teraz zajmiemy się rozwiązaniem docelowym, czyli takim, które będzie wykorzystywało $O(m)$ komórek pamięci, gdzie $m$ jest związane z licznością podzbioru $U$, który chcemy przechowywać w słowniku.

W tym celu weźmy funkcję $h : U \rightarrow \{0, 1, \dots, m - 1\}$. Każdą funkcję tego typu nazywamy funkcją haszującą, jednak nie każda z nich jest tak samo dobra, dlatego zdefiniujmy \textit{dobrą} funkcję haszującą.
\begin{definition}[Dobra funkcja haszująca]
Funkcję haszującą $h$ nazwiemy \textit{dobrą}, gdy:
\begin{equation*}
\forall_{j = 0, \dots, m - 1} \sum_{k : h(k) = j} Pr(k) = \dfrac{1}{m} \wedge h \textsl{ jest łatwo obliczalna}
\end{equation*}
gdzie $Pr(k)$ jest prawdopodobieństwem tego, że $k \in U$ jest argumentem którejś z operacji wykonywanych na słowniku.
\end{definition}
W praktyce powyższy warunek jest niesprawdzalny, ponieważ nie znamy $Pr(k)$.
Przykłady funkcji haszujących:
\begin{itemize}
	\item $h(k) = k \mod m$, gdzie $m$ nie powinno być postaci $2^p$ (ponieważ $h(k)$ jest $p$ ostatnimi bitami $k$. Dużo lepsze wyniki dają $m$ będące liczbami pierwszymi oddalonymi od potęg dwójki.
	\item $h(k) = \lfloor m(kA - \lfloor kA \rfloor) \rfloor$, gdzie $A \in (0, 1)$, $m$ może być potęgą dwójki.
\end{itemize}


%% Dodatki

\begin{appendices}

\pagestyle{empty}
\input{porownanie.tex}

\end{appendices}

\end{document}
